\section{FPGA}
FPGA are rather preferred over DSPs for digital signal processing when a high sample rate is needed. A normal DSP is designed to perform a specific task but can have limitations in the number of tasks that can be done per second. In today’s world, demands are increasing faster than the development of DSPs which means in many cases a single DSP is no longer efficient. One solution is to use multiple DSPs but architecture is more complex and an implementation of this type is not cost efficient. Using multiple devices forces the programmer to code some scheduling algorithms in order to achieve what is wanted which adds more overhead on the machine. 
In order to further understand why an FPGA is more efficient than the traditional DSP, the two modules need to be further explained.

\subsection{How does the DSP work?}