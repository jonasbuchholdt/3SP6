\section{FPGA}


The FPGA (Field Programmable Gate Array) is an integrated circuits that contains multiple logic blocks and routing channels. \\
The hardware configuration is customizable in order to meet the user’s demands; unlike ICs where the connections between the transistors cannot be changed, it is possible to make a custom architecture on an FPGA. \\
The biggest advantage of the FPGA is to make multiple processes at the same time, which is not the case in the ICs. If task needs to be done in a restricted timing, this parameter can be crucial. 
As said, the FPGA consists of multiple logic blocks often called logic cells. Each cells consists of an LUT (Look-up table) and a flip-flop. The LUT has four inputs and the flip-flop one clock input. The LUT contains a small Random Access Memory and performs logical operation. \\


The FPGA can be programmed using a hardware description language like VHDL. \\







