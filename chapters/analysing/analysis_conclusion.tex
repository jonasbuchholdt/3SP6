In the previous \autoref{ch:analysing}, different sound personalizations that can be done without any effect pedals on an electric guitar has been scrutinized. The electronics in an electric guitar have been described, the output impedance and frequency spectrum of an electric guitar have been measured. Then, different effects have been presented followed by the advantages of digital signal processing over analog. Different devices for \gls{dsp} have also been presented.  \\
\newline
It can be inferred from the analysis \autoref{sec:electric_guitar_theory} that the level of customization when using the physical part of the guitar is very low. However, there is a great number of effects that can be attached to this type of guitars and give the musician more possibilities, some of them presented in \autoref{sec:effects}.  \\
During the last decades, a lot of improvements have been made in the digital domain. The digital sound quality is now comparable to the analog one even for an audiophile. The advantages of using analog signal processing over digital are getting weaker. On the other side, various reasons can push the user to choose digital over analog: portability, lifetime, personalization, ease of use...
