In the previous \autoref{ch:analysing}, different sound personalizations that can be done without any effect pedals on an electric guitar have been scrutinized. The electronics in an electric guitar have been described, the output impedance and frequency spectrum of an electric guitar have been measured. Then, different effects have been presented followed by the advantages of digital signal processing over analog. Different devices for digital signal processing have also been presented. 
It can be inferred from the analysis \autoref{sec:electric_guitar_theory} that the level of customization when using the physical part of the guitar is very low. However, there is a great number of effects that can be attached to this type of guitars and give the musician more possibilities, some of them presented in \autoref{sec:effects}. Therefore it is chosen to focus on the effects, instead of the physical parts on the the guitar, in the rest of the project.  \\
During the last decades, a lot of improvements have been made in the digital domain. The digital sound quality is now comparable to the analog one even for an audiophile. It is there chosen to work with the effects digitally.
