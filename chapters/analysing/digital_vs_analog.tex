\section{The benefits of using digital signal processing over analog}
\subsection{Quality}

The bandwidth in an analog sound signal can be increased infinitely unlike the digital one where it is limited by the sampling rate that is being used. \\

The quality in the digital signal is  lower because in the digital domain there is what is called the digital noise. It occurs after sampling, and increase when the sound between samples get less and less smooth. On the other side, in the analog domain, the signal is not sampled, it is the real one so this problem do not exist. Errors can also occur due to bit depth, bit depth limits the values that the sample can take, which means that even with high sample rate but low bit depth, the digital track can sound different. The number of values that can be sampled is defined by the number of bits per sample, and the number of values are then $2^{number of bits}$ \citep{analog_quality}.\\
It can then be inferred that there are quantization noise and errors in digital signals, which do not exist in analog signals. The same goes for aliasing \citep{analog_aliasing}. 

\subsection{Mobility}

Digital Sounds are more portable than analog ones, they can be played on different machines and copied without losing any information unlike the analog sound that can only be played on a tape for instance \citep{analog_quality}. 

\subsection{Lifetime}

A digital sound can last longer than an analog one. A Vinyl disk for example can get damaged and impact the sound stored on it and thus the quality. On the other side, a digital file saved on a computer cannot get damaged physically, unless the storage does, and can be duplicated very easily to preserve it. With the rise of cloud systems, the loss of a digital file is nearly impossible unless intentional \citep{analog_storage}.


\subsection{Requirements}

It is easier to fulfil given requirements using digital over analog signal processing because the user has more control over the digital controller than the analog one. If at anytime, the engineer realizes that the components needs more processing power, it can be easily upgraded \citep{analog_requirements}.

\subsection{Functionality and Customization}

With only one \gls{dsp}, it is possible to program numerous effects, personalize them and even create custom ones. Effects presented in \autoref{sec:effects} are just part of a growing and endless world of digital effects. 

 