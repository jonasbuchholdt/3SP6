\section{Cost}


\section{Quality}

The bandwidth in an analog sound signal can be increased infinitely unlike the digital one where it is limited by the sampling rate that has been used. \\
The \gls{snr} is higher in digital signals than analog ones because of their low bit depth. The bit depth is a variable that takes into account the change in amplitude between each sample. If the sample rate is not high enough in digital signal, the \gls{snr} is higher \citep{analog_quality}. \\
Some will argue that by sufficiently increasing the sampling rate, the quality difference between analog and digital sounds can no longer be felt \citep{analog_storage}.

\section{Mobility}

Digital Sounds are more portable than analog ones, they can be played on different machines and copied without losing any information unlike the analog sound that can only be played on a tape for instance \citep{analog_quality}. 

\section{Lifetime}

A digital sound can last longer than an analog one. A Vinyl disk for example can get damaged and impact the sound stored on it and thus the quality. On the other side, a digital file saved on a computer cannot get damaged physically, unless the storage does, and can be duplicated very easily to preserve it. With the rise of cloud systems, the loss of a digital file is nearly impossible unless intentional \citep{analog_storage}. 