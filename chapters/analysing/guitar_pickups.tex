\section{Pickups on an Electric Guitar}\label{sec:pickups}
One of the most important parts of an electric guitar, is the transducer which converts the strings vibrations into an electrical signal. This is done with pickups, which as seen in \autoref{fig:guitar_parts}, is placed on the body. Typically an electric guitar has two or three pickups and the type and combination of pickups can vary. Two types of pickups are the single coil pickups and the dual coil (Humbucker) pickups. The electric equivalent to a single coil pickup is shown in \autoref{fig:electric_pickup}

\begin{figure}[h!]
\centering
\begin{circuitikz}\draw (0,0)
to[L=$L$]  (3,0)
to[R=$R$] (5,0)
(5,-1)to[short,-o](6,-1)
(5,0)to[short](5,-2)
to[C=$C$] (0,-2)
(0,-1)to[short,-o](-1,-1)
(0,-2)to[short] (0,0)
%to[short] (0,0)
;\end{circuitikz}
\caption{Electric equivalent of a single coil pickup \citep{build_your_guitar}.}
\label{fig:electric_pickup}
\end{figure}

A single coil pickup can be described as an ideal inductor and a resistor in series, in parallel with a capacitor. This circuit is for the case when the guitars strings do not vibrate. When one or more strings are played, the circuit is expanded as shown in \autoref{fig:electric_pickup_strings}.

\begin{figure}[h!]
\centering
\begin{circuitikz}\draw (0,0)
to[L=$L$]  (3,0)
to[R=$R$] (5,0)
to[short, -o](6,0)
(5,0)to[C=$C$] (5,-2)
(6,-2)to[short, o-] (0,-2)
to[sV=$String$] (0,0)
%to[short] (0,0)
;\end{circuitikz}
\caption{Electric equivalent of a single coil pickup, when strings are played \citep{build_your_guitar}.}
\label{fig:electric_pickup_strings}
\end{figure}

When the strings vibrate, they act as an \gls{ac} supply. The output signal from the pickup is then the voltage over the capacitor  \citep{build_your_guitar}. 
The complete circuit in an electric guitar is a bit more complicated, since an electric guitar often contains a volume control, one or more tone controls etc. A simple circuit of an electric guitar with one volume control and one tone control, is shown in \autoref{fig:simple_guitar_circuit}.
\begin{figure}[h!]
\centering
\begin{circuitikz}\draw (0,0)
to[L=$Pickup$]  (0,4)
to[short] (2,4)
to[vR=$Tone$ $pot$] (2,2)
to[C=$Tone$ $C$] (2,0)
to[short](0,0)
(2,4) to[short](5,4)
to[pR=$ $](5,0)
to[short](2,0)node[ground]{}
%Jack
(5.3,2)to[short]node[left,above]{$Volume$ $pot$}(6.6,2)
to[short](6.9,1.7)
to[short](7.2,2)
(7.5,2)to[short](7.5,1)
to[short](6,1)
to[short](6,0)
to[short](5,0)
(7.5,2)to[short](7.7,2)
to[short]node[right]{$Jack$ $input$}(7.7,1)
to[short](7.5,1)
%end Jack
;\end{circuitikz}
\caption{Simple circuit of an electric guitar with one volume control and one tone control \citep{electricalfun}.}
\label{fig:simple_guitar_circuit}
\end{figure}

The tone control is a simple lowpass filter and the volume control is a potentiometer. 

\subsection{Output Impedance of an Electric Guitar}
A test was made to find the output impedance of an electric guitar. The test was made on a Fender Squier Classic Vibe Telecaster, and was tested with three pickup settings. The complete test journal can be seen in \autoref{app:output_impedance}. In \autoref{tab:impedance_test} the results from the output impedance test on the guitar is shown.

\begin{longtable}[h!]{ |m{\dimexpr 0.34\linewidth-2\tabcolsep}| 
          m{\dimexpr 0.33\linewidth-2\tabcolsep}| 
          m{\dimexpr 0.33\linewidth-2\tabcolsep}|   } 
\caption{Results from output impedance test of an electric guitar.} \label{tab:impedance_test} \\ 
 
\hline 
%%%%%%%%  
\textbf{Pickup setting} & \textbf{Minimum output impedance} & \textbf{Maximum output impedance} \\ 
%%%%%%%% 
\hline 
\endfirsthead     
\multicolumn{3}{c}{{{\footnotesize \bfseries \tablename\ \thetable{} -- Continued from previous page.}}} \\  
\hline 
%%%%%%%%  
\textbf{Pickup setting} & \textbf{Minimum output impedance} & \textbf{Maximum output impedance} \\ 
%%%%%%%%  
\hline 
\endhead       
\hline \multicolumn{3}{|r|}{{Continues on next page.}} \\ \hline 
\endfoot     
\hline 
\endlastfoot 
%%%%%%%% Content of table 
Neck & \SI{6220}{\ohm} at \SI{10}{\hertz} & \SI{58,57}{\kilo\ohm} at \SI{5104}{\hertz} \\ \hline
Bridge & \SI{7863}{\ohm} at \SI{10}{\hertz}  & \SI{73.08}{\kilo\ohm} at \SI{4230}{\hertz}\\ \hline
Neck and bridge & \SI{3563}{\ohm} at \SI{10}{\hertz} & \SI{53.43}{\kilo\ohm} at \SI{5702}{\hertz}\\ \hline
\end{longtable}

\subsection{Frequency Area of an Electric Guitar}
A test was made to see which frequencies can be played on an electric guitar. The test was made with a Fender Squier Classic Vibe Telecaster, and was tested with two pickup settings. The complete test journal can be seen in \autoref{app:frequency_area}. In \autoref{tab:frequency_area} the results from the frequency spectrum test on the guitar is shown.

\begin{longtable}[h!]{ |m{\dimexpr 0.34\linewidth-2\tabcolsep}| 
          m{\dimexpr 0.33\linewidth-2\tabcolsep}| 
          m{\dimexpr 0.33\linewidth-2\tabcolsep}|   } 
\caption{Results from output impedance test of an electric guitar.} \label{tab:frequency_area} \\ 
 
\hline 
%%%%%%%%  
\textbf{Pickup setting} & \textbf{Lowest significant frequency} & \textbf{Highest significant frequency} \\ 
%%%%%%%% 
\hline 
\endfirsthead     
\multicolumn{3}{c}{{{\footnotesize \bfseries \tablename\ \thetable{} -- Continued from previous page.}}} \\  
\hline 
%%%%%%%%  
\textbf{Pickup setting} & \textbf{Lowest significant frequency} & \textbf{Highest significant frequency} \\ 
%%%%%%%%  
\hline 
\endhead       
\hline \multicolumn{3}{|r|}{{Continues on next page.}} \\ \hline 
\endfoot     
\hline 
\endlastfoot 
%%%%%%%% Content of table 
Neck & Around \SI{80}{\hertz} & Around \SI{4400}{\hertz} \\ \hline
Bridge & Around \SI{80}{\hertz}  & Around \SI{4400}{\hertz}\\ \hline
\end{longtable}

\subsection{Maximum and Minimum Output of an Electric Guitar}
A test was made to measure the maximum- and the minimum- output of an electric guitar. The test was made with a Fender Squier Classic Vibe Telecaster, and was made with the bridge pickup active. The complete test journal can be seen in \autoref{app:guitar_max_amplitude}. The maximum output was measured to be $\SI{1}{\volt}_{peak}$.
In \autoref{app:frequency_area} \autoref{fig:appendix:high_E_bridge_flasholet} it is seen that the smallest significant signal is \SI{-59.11}{\deci \bel u}. To convert \SI{}{\deci \bel u} to amplitude \autoref{eq:dBu_to_amplitude} is used \citep{dB}.

\begin{subequations}
\begin{equation}\label{eq:V_RMS_to_dBu}
        dBu = 20 \cdot \log_{10}\left(\frac{V_{RMS}}{0.775}\right)
    \end{equation}
\centering
$\Updownarrow$
\begin{equation}\label{eq:dBu_to_amplitude}
        V_{peak} = \sqrt{2} \cdot 0.775 \cdot 10^{\frac{dBu}{20}}
    \end{equation}
 \end{subequations}
 
When inserting \SI{-59.11}{\deci \bel u} in \autoref{eq:dBu_to_amplitude}, the amplitude of the smallest signal can be calculated as in \autoref{eq:dBu_to_amplitude_number}.

\begin{equation}\label{eq:dBu_to_amplitude_number}
       \sqrt{2} \cdot 0.775 \cdot 10^{\frac{-59.11}{20}} = 1.21 \addunit{$\SI{}{\milli \volt}_{peak}$}
    \end{equation}

The relation between the maximum and the minimum signal in dB is calculated in \autoref{eq:dB_max_min_guitar_out}

\begin{equation}\label{eq:dB_max_min_guitar_out}
       20 \cdot \log_{10}\left(\frac{1}{0.00121}\right) = 58.34 \addunit{$\SI{}{\deci \bel}$}
    \end{equation}



