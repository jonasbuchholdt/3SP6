\subsection{Reverberation}
\gls{reverb} is an effect where a sound wave is reflected, and therefore those waves have another traveling time and amplitude than the initial sound directly from the source  when they arrive to the listener. This effect is very frequent; it happens each time a sound is reflected by walls, trees, table and other surfaces which reflect sound. The following \autoref{fig:reverb_reflect} shows  \gls{reverb} effect waves paths from one person, the sound source to another , the listener \citep{reverb_expl}.

\begin{figure} [htbp]
 \centering
\begin{picture}(0,0)%
\includegraphics{reverb_reflect.pdf}%
\end{picture}%
\setlength{\unitlength}{4144sp}%
%
\begingroup\makeatletter\ifx\SetFigFont\undefined%
\gdef\SetFigFont#1#2#3#4#5{%
  \reset@font\fontsize{#1}{#2pt}%
  \fontfamily{#3}\fontseries{#4}\fontshape{#5}%
  \selectfont}%
\fi\endgroup%
\begin{picture}(4884,2544)(2329,-3313)
\put(2746,-1006){Sound wave}%
\put(2746,-2086){$Source$}%
\put(6031,-2086){$Listener$}%
\end{picture}%
  \caption{The photo shows a echo in time domain}
  \label{fig:reverb_reflect}
\end{figure}

The received reflected sound is actually a mix of very fast echoes, a merge with all the reflected sounds and the direct one, so the effect is noticed as a single one, although that may be more than 100 echoes. 
\gls{reverb} makes a complex mix of echoes which brings depth to the guitar sound, and makes it more natural \citep{reverb_natural}

A simple block diagram is shown at \autoref{fig:reverb_block}.

\begin{figure} [htbp]
 \centering
\begin{picture}(0,0)%
\includegraphics{reverb.pdf}%
\end{picture}%
\setlength{\unitlength}{4144sp}%
\begingroup\makeatletter\ifx\SetFigFont\undefined%
\gdef\SetFigFont#1#2#3#4#5{%
  \reset@font\fontsize{#1}{#2pt}%
  \fontfamily{#3}\fontseries{#4}\fontshape{#5}%
  \selectfont}%
\fi\endgroup%
\begin{picture}(5787,3102)(886,-1603)
\put(5896,-286){$Output$}%
\put(4321,704){$Gain$}%
\put(2791,1334){Initial sound directly from the source}%
\put(901,1334){$Input$}%
\put(2791,434){$Delay$}%
\put(2791,-466){$Delay$}%
\put(2791,-1366){$Delay$}%
\put(4321,-241){$Gain$}%
\put(4321,-1141){$Gain$}%
\end{picture}%
  \caption{The photo shows a block diagram on a \gls{reverb} unit}
  \label{fig:reverb_block}
\end{figure}

The block diagram \autoref{fig:reverb_block} shows the direct sound way and only a few delay blocks with their individual gain. Normally, there are many more delay lines. All signals are added together just before the Output line.