\section{Wah-Wah}

The effect takes the original signal and mix it with another signal that passes through a bandpass filter. The bandpass filter is time varying, which means that it changes its position in the frequency spectrum. \\
Different parameters can be changed on the \gls{lfo} to customize the effect:\\

\begin{itemize}
	\item \textbf{The \gls{lfo} frequency}: it sets the speed at which the bandpass filter moves in the frequency spectrum.
	\item \textbf{\gls{lfo} start phase}: Determine where should the bandpass filter start.
	\item \textbf{\gls{lfo} depth}: the range of frequencies it should work on, high depth gives a bigger range and vice versa.
\end{itemize}

A block diagram of the effect is illustrated in \autoref{wah-wah-diag}.  

\begin{figure} [htbp]
	\centering
	\includegraphics[width=1\textwidth]{wah-wah-diag.png}
	\caption{Block diagram of the wah-wah effect}
	\label{wah-wah-diag}
\end{figure}

The phaser effect can be created by using a band-stop filter instead of a bandpass filter using the same block diagram presented in \autoref{wah-wah-diag}.

There is another type of wah-wah effect called M-fold wah-wah which uses multiple M-tap bandpass filters that move around the spectrum at the same time.
