\chapter{Flanger and Chorus MATLAB Code and Simulation}

The Matlab code for the flanger effect is:

\begin{lstlisting}[language=Matlab, caption= Matlab code for flanger effect]

%Flanger Effect
%Group 641
%Mohamed Gabr, Jonas and Sebastian

clear all;
filename = 'guitartune.wav'; 
%The filename is inserted here, should be in the same directory as the
%codefile.
[input, fs] = audioread(filename);
%input samples and sample rate are extracted from the file.
input = input / (max(abs(input)));
%the input is rescaled in order to have values between -1 and 1. 
fl = 1;  
%LFO Frequency is chosen here.
sample_no = length(input); 
%Length of the input samples table.
g = 0.7; 
%Gain
delay_s = 0.003; 
% The maximum delay in seconds.
after_delay = (1:1:sample_no); 
before_delay = (1:1:sample_no);
output = (1:1:sample_no);
%Needed tables in the code are intialized to make the code faster.
disp('max delay')
max_delay = round(delay_s * fs)  
%Convert the maximum delay from delay in seconds to delay in samples.

buffer = zeros(1,abs(round(max_delay))); 
%intializing the buffer array.



for i = 1:1:sample_no  %Loop that treats a sample per iteration. 
	delay = (max_delay) * abs(sin(2*pi*i*(fl/fs))) - 2;  %Calculate time varying delay 
	%(unit is samples)
	
	buffer = [buffer(2:end) input(i)]; 
	%move the values in the buffer ==> first value overwritten, one new 
	%value added at the end.
	if (max_delay - floor(abs(delay))) == 0  %All delay values less than zero are considered as
	    %not having a delay
	        after_delay(i) = buffer(1);
	else
	    if floor(abs(delay)) == abs(delay)
	            after_delay(i) = buffer(max_delay-floor(abs(delay))) * g;
	    %Treating the case when the delay is an integer and when it has a
	    %value bigger than the maximum delay. 
	    else %Treating the case when the delay is not an integer
	        x1 = max_delay - floor(abs(delay));
	        x2 = x1 - 1;
	        y1 = buffer(x1);
	        y2 = buffer(x2);
	        d1 = y2 - y1;
	        c1 = x2 - x1;
	        c2 = max_delay - abs(delay) - x1;
	        d2 = (c2/c1) * d1;
	        final = y1 + d2;
	
	        %Making a linear approximation between the samples to increase
	        %precision
	        after_delay(i) = final * g;  %sound from the delay line
	        %in the block diagram 
	    end
	end
	before_delay(i) = input(i) * g ;  %sound from the direct line
	output(i) = after_delay(i) * g + before_delay(i); %output delayed signal 
end
output = output / max(abs(output)); %output is rescaled to values between
% -1 and 1 to avoid clipping. 
audiowrite('acoustic_out.wav',output,fs); %Writing the values in a file, 
% PS: audiowrite will clip if the values are not between -1 and 1. 
plot(output,'r') %plotting the output samples in red.
hold on
grid
plot(before_delay,'b') %plotting the input samples in blue on the same fig.






\end{lstlisting}


The code is commented for the user's instruction.\\

The Matlab code for the chorus effect is:

\begin{lstlisting}[language=Matlab, caption= Matlab code for chorus effect]

%Chorus Effect
%Group 641
%Mohamed, Jonas and Sebastian
clear all;
filename = 'guitartune.wav'; %Typing the filename, should be in same 
%directory as the codefile
[input, fs] = audioread(filename);
%input samples and sample rate are extracted from the file.
input = input / (max(abs(input)));
%the input is rescaled in order to have values between -1 and 1.
lines = 4; %number of chorus lines 
fl = randi(1,1,lines);
%LFO Frequencies for each line are generated.
delay = (1:1:lines);
%Delay table generated here to speed the coding. 
sample_no = length(input); 
%Length of the input samples
g = rand(1,lines);
%Gain values generated here
delay_s = 0.0025; 
% Maximum delay in seconds 
after_delay = rand(lines, sample_no); 
before_delay = (1:1:sample_no);
output = (1:1:sample_no);
% Generating tables to speed the code. 
max_delay = round(delay_s * fs)  
%convert the delay from delay in seconds to delay in samples. 


buffer = zeros(1,abs(round(max_delay))); %create the buffer array 
    
for i = 1:1:sample_no %loop that gives one output sample per iteration
    buffer = [buffer(2:end) input(i)]; %moving buffer values
    before_delay(i) = input(i) * g(1); %Value before delay
    output(i) = before_delay(i); %adding it to one output sample
    for j = 1:1:lines %Generating delays and then samples after delay for 
        %each chorus line
        delay(j) = max_delay * abs(sin(2*pi*i*(fl(j)/fs))) - 1; %Calculate time  
    %varying delay (unit is samples)
        if (max_delay - floor(abs(delay(j)))) == 0 %Treating the case where the delay is less 
            %than one.
                after_delay(j,i) = buffer(1);
        else
            if floor(delay(j)) == abs(delay(j)) || abs(floor(delay(j))) >= floor(max_delay)
                    after_delay(j,i) = buffer(max_delay - floor(abs(delay(j)))) * g(j);
          %Treating the case when the delay is an integer and when it has a
          %value bigger than the maximum delay. 
            else
            x1 = max_delay - floor(abs(delay(j)));
            x2 = x1 - 1;
            y1 = buffer(x1);
            y2 = buffer(x2);
            d1 = y2 - y1;
            c1 = x2 - x1;
            c2 = max_delay - abs(delay(j)) - x1;
            d2 = (c2/c1) * d1;
            final = y1 + d2;
            %Making a linear approximation between the samples to increase
            %precision
            after_delay(j,i) = final * g(j);  
            %sound from the delay line in the block diagram
            
            end
        end
        output(i) = output(i) + after_delay(j,i);
        %Signal output sample
    end
    
end
output = output / max(abs(output)); %output is rescaled to values between
% -1 and 1 to avoid clipping. 
audiowrite('out_chor.wav',output,fs); %Writing the values in a file, 
% PS: audiowrite will clip if the values are not between -1 and 1.
plot(output,'r') %plotting the output samples in red.
hold on
grid
plot(before_delay,'b') %plotting the input samples in blue on the same fig.



\end{lstlisting}

The codes where simulated on different audiofiles. 