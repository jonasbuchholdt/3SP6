\chapter{Flanger and Chorus MATLAB code and simulation}x

The Matlab code for the flanger effect is:

\begin{lstlisting}[language=Matlab, caption= Matlab code for flanger effect]

%Flanger Effect
%Group 641
%Mohamed Gabr, Jonas and Sebastian


filename = 'inputttt.wav'; 
%The filename is inserted here, should be in the same directory as the
%codefile.
[input, fs] = audioread(filename, 'native');
%input samples and sample rate are extracted from the file.
input = input / (max(abs(input)));
%the input is rescaled in order to have values between -1 and 1. 
fl = 100;  
%LFO Frequency is chosen here.
sample_no = length(input); 
%Length of the input samples table.
g = 100; 
%Gain
delay_s = 0.0050; 
% The maximum delay in seconds.
after_delay = (1:1:sample_no); 
before_delay = (1:1:sample_no);
output = (1:1:sample_no);
%Needed tables in the code are intialized to make the code faster.
disp('max delay')
max_delay = delay_s * fs  
%Convert the maximum delay from delay in seconds to delay in samples.

buffer = zeros(1,abs(round(max_delay))); 
%intializing the buffer array.

for i = 1:1:sample_no  %Loop that treats a sample per iteration. 
 delay = max_delay * cos(2*pi*i*(fl/fs)); %Calculate time varying delay 
 %(unit is samples)
 buffer = [buffer(2:end) input(i)]; 
 %move the values in the buffer ==> first value overwritten, one new 
 %value added at the end.
 if abs(delay) < 1   %All delay values less than zero are considered as
  %not having a delay
  after_delay(i) = 0;
 else
  if floor(delay) == delay || abs(floor(delay)) >= floor(max_delay)
   after_delay(i) = buffer(round(abs(delay))) * g;
   %Treating the case when the delay is an integer and when it has a
   %value bigger than the maximum delay. 
  else %Treating the case when the delay is not an integer
   x1 = abs(floor(delay));
   x2 = x1 + 1;
   y1 = buffer(x1);
   y2 = buffer(x2);
   coeff = (y1 - y2)/(x1 - x2);
   %Making a linear approximation between the samples to increase
   %precision
   after_delay(i) = delay * coeff * g;  %sound from the delay line
   %in the block diagram 
  end
 end
 before_delay(i) = input(i);  %sound from the direct line
 output(i) = after_delay(i) + before_delay(i); %output delayed signal 
end
output = output / max(abs(output)); %output is rescaled to values between
% -1 and 1 to avoid clipping. 
audiowrite('out_flan.wav',output,fs); %Writing the values in a file, 
% PS: audiowrite will clip if the values are not between -1 and 1. 
plot(output,'r') %plotting the output samples in red.
hold on
grid
plot(before_delay,'b') %plotting the input samples in blue on the same fig.




\end{lstlisting}


The code is commented for the user's instruction.\\

The Matlab code for the chorus effect is:

\begin{lstlisting}[language=Matlab, caption= Matlab code for chorus effect]

%Chorus Effect
%Group 641
%Mohamed, Jonas and Sebastian

filename = 'inputttt.wav'; %Typing the filename, should be in same 
%directory as the codefile
[input, fs] = audioread(filename);
%input samples and sample rate are extracted from the file.
input = input / (max(abs(input)));
%the input is rescaled in order to have values between -1 and 1.
lines = 5; %number of chorus lines 
fl = randi(10,1,lines);
%LFO Frequencies for each line are generated.
delay = (1:1:lines);
%Delay table generated here to speed the coding. 
sample_no = length(input); 
%Length of the input samples
g = rand(1,lines); 
%Gain values generated here
delay_s = 0.025; 
% Maximum delay in seconds 
after_delay = rand(lines, sample_no); 
before_delay = (1:1:sample_no);
output = (1:1:sample_no);
% Generating tables to speed the code. 
max_delay = delay_s * fs  
%convert the delay from delay in seconds to delay in samples. 


buffer = zeros(1,abs(round(max_delay))); %create the buffer array 

for i = 1:1:sample_no %loop that gives one output sample per iteration
 buffer = [buffer(2:end) input(i)]; %moving buffer values
 before_delay(i) = input(i); %Value before delay
 output(i) = before_delay(i); %adding it to one output sample
 for j = 1:1:lines %Generating delays and then samples after delay for 
 %each chorus line
   delay(j) = max_delay * cos(2*pi*i*(fl(j)/fs)); %Calculate time  
   %varying delay (unit is samples)
   if abs(delay(j)) < 1 %Treating the case where the delay is less 
   %than one.
	 after_delay(j,i) =0;
   else
	 if floor(delay(j)) == delay(j) || abs(floor(delay(j))) >= floor(max_delay)
	   after_delay(j,i) = buffer(round(abs(delay(j)))) * g(j);
	   %Treating the case when the delay is an integer and when it has a
	   %value bigger than the maximum delay. 
	 else  %Treating the case when the delay is not an integer
	   x1 = abs(floor(delay(j)));
	   x2 = x1 + 1;
	   y1 = buffer(x1);
	   y2 = buffer(x2);
	   coeff = (y1 - y2)/(x1 - x2);
	   %Making a linear approximation between the samples to increase
	   %precision
	   after_delay(j,i) = delay(j) * coeff * g(j);  
	   %sound from the delay line in the block diagram
	 end
   end
   output(i) = output(i) + after_delay(j,i);
   %Signal output sample
 end
end
output = output / max(abs(output)); %output is rescaled to values between
% -1 and 1 to avoid clipping. 
audiowrite('out_chor.wav',output,fs); %Writing the values in a file, 
% PS: audiowrite will clip if the values are not between -1 and 1.
plot(output,'r') %plotting the output samples in red.
hold on
grid
plot(before_delay,'b') %plotting the input samples in blue on the same fig.



\end{lstlisting}

The codes where simulated on different audiofiles. 