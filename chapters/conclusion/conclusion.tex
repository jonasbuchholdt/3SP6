\section{Conclusion}\label{sec:conclusion}

In the analysis \autoref{ch:analysing}, some initial problems were answered, regarding how the sound from an electric guitar could be personalized. It was found that a way to personalize the sound was by sending the signal through guitar effects, and it was found that by making the effects digitally, certain advantages followed. The following problem statement was made from this research:

\textbf{How can a number of digital guitar effects be designed and implemented using Digital Signal Processing?} \\

24 requirements were made for the further development of the product. 
It was found that the use of a \gls{dsp} was an ideal choice for implementing digital guitar effects. A \gls{preamp} was designed and constructed to connect the guitar and the \gls{dsp}, the \gls{dsp} was explained, and the initial setup of the \gls{dsp} was made. The five effects; the equalizer, \gls{reverb}, delay, chorus and flanger where designed, and the three effects; the equalizer, \gls{reverb} and flanger were implemented on the \gls{dsp} using asembly.
The final system were tested and held against the stated requirements. \\

It can be concluded that parts of the equalizer, the \gls{reverb} and the flanger could be implemented digitally on a \gls{dsp}.  


