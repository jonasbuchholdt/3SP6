\graphicspath{{figures/design/}}
\chapter{Modular design}\label{ch:moddesign}

To facilitate a parallel design process the robotic lawn mower is divided into modules. Each module can be constructed and tested individually which should allow for a plug and play connection of the finished modules. The division of the modules with their roles is shown in \autoref{fig:modular_design}. The composition of each module is based on the previous chapter: \textit{Evaluation of possible solutions} and contains a series of sub-modules with more specific functionality. The following chapters will describe the design and construction of each module with its sub-modules. It is only sub-modules marked with grey which is developed through this project.

\begin{figure}[htb]
\includegraphics[width=\textwidth]{Moduldiagram}
\caption{Modular design of the robotic lawn mower.}
\label{fig:modular_design}
\end{figure}

All modules in \autoref{fig:modular_design} consist of a combination of hardware and software. The \textit{Basic Services} is, as the name suggests, a module which provides services to other modules. The module \textit{Route Planning} is responsible for planning a route to mow a lawn. \textit{Drive Handling} is the execution of the route via commands to the drive train and an obstacle detection system. The module \textit{Drive Train} covers the drive train, with motors, gears and wheels, including the \gls{pwm} generator and a drive train controller. The module \textit{User Interface} has a general system to handle user request and two user interfaces; a remote and a local. 
