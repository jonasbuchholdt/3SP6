\section{Driving controller}

The driving controller must control the angular velocity of the motors using feedback from the \gls{dgps}. To simplify the calculations revolving the modelling of the lawn mower, the vehicle be modelled as having two wheels instead of crawler treads.
\autoref{fig:motor_calc2} shows the two wheels of the vehicle, placed in a coordinate system.

\begin{figure}[h]
\centering
\includegraphics[width=0.6\textwidth]{motor2}
\caption{Vehicle to be turned towards reference point $x_{ref}$.}
\label{fig:motor_calc2}
\end{figure}

The center point of the vehicle is the current position received from the \gls{dgps}. The direction of the vehicle is determined as seen in \autoref{eq:directionvect}.

\begin{equation}
\vec{F} = \vec{x_c} - \vec{x_{c,prev}}  \addunit{}
\end{equation}
\startexplain
	\explain{$\vec{F}$ is the direction of the vehicle}{}
	\explain{$x_{c}$ is the center point of the vehicle}{}
	\explain{$x_{c,prev}$ is the previous position}{}
	\label{eq:directionvect}
\stopexplain

To get the vehicle rotated to follow the vector $\vec{V_T}$, a reference point $x_{ref}$ must be calculated, to determine whether there is an angle between the direction vector $\vec{F}$ and the reference vector $\vec{V_{ref}}$.
The reference point $x_{ref}$ is determined as a predefined distance from $x_{ort}$, which is a orthogonal point from the center point of the vehicle, and determined as:

\begin{equation}
x_{ort} = \frac{\vec{V_T}\bullet (x_c - x_0)}{|\vec{V_T}|^2} \cdot \vec{V_T} + x_0 \addunit{}
\end{equation}
\startexplain
		\explain{$x_{ort}$ is the orthogonal point from the center point to $V_T$}{}
		\explain{$x_0$  is the previous point at the route}{}
		\explain{$V_T$ is the vector from the previous point to the next point on the route}{}
		\explain{$x_c$ is the center point of the vehicle}{}
\stopexplain
 
The reference point $x_{ref}$ can then be determined as:

\begin{equation}
x_{ref} = x_{ort} + \Delta t\cdot \vec{V_T} \addunit{}
\end{equation}

By this, the reference vector $\vec{V_{ref}}$ can be determined as:

\begin{equation}
\vec{V_{ref}} = \vec{x_c} - \vec{x_{ref}} \addunit{}
\end{equation}

By calculating the cross product between the direction vector $\vec{F}$ and the reference vector $\vec{V_{ref}}$, it can be determined whether the angle is different from $0$.

The difference between the angular velocities of the two motors can be determined as:

\begin{equation}
\omega_d = -K \cdot \frac{\vec{V_{ref}}\times \vec{F}}{|\vec{V_{ref}}|} \addunit{\SI{}{\radian/\second}}
\end{equation}

With this difference, the motor velocities can by a predefinition of the sum of the two motor velocities and the equations:

\begin{equation}
\omega_d = \omega_L - \omega_R \addunit{\radian\per\second}
\end{equation}
\begin{equation}
\omega_s = \omega_L + \omega_R \addunit{\radian\per\second}
\end{equation}
be determined as 
\begin{equation}
\omega_L = \frac{\omega_d}{2} + \frac{\omega_s}{2} \addunit{\radian\per\second}
\end{equation}
\begin{equation}
\omega_R = -\frac{\omega_d}{2} + \frac{\omega_s}{2} \addunit{\radian\per\second}
\end{equation}

\subsection{Simulation of driving}
In order to validate whether or not, the algorithms are able to control the motors, a model of the drive controller is created and simulated in MATLAB. Derivation for some of the following equations are not written out in this section, instead they are located in \autoref{app:drive-equations}.

\autoref{fig:motor_calc} shows the two wheels of the vehicle, placed in a coordinate system.

\begin{figure}[h]
\centering
\includegraphics[width=0.6\textwidth]{motor}
\caption{Vehicle position and direction.}
\label{fig:motor_calc}
\end{figure}

To determine the direction of the vehicle in the simulation, the vector $\vec{L}$ from the left wheel to the right, is rotated $\SI{\pi /2}{rad}$.
The $\vec{L}$ vector can be determined as:

\begin{equation}
\vec{L}=\vec{X_L}-\vec{X_R} \addunit{}
\end{equation}
\startexplain
		\explain{$\vec{X_L}$  is the vector from origin to the left wheel}{}
		\explain{$\vec{X_R}$  is the vector from origin to the right wheel}{}
	\stopexplain
	 
A unit vector $\vec{U_L}$ of $\vec{L}$ is determined as:

\begin{equation}
\vec{U_L}=\frac{\vec{L}}{|\vec{L}|} \addunit{}
\end{equation}

By creating a rotational matrix $R$ rotated $\SI{\pi /2}{rad}$ as:

\begin{equation}
R= \begin{bmatrix}
	cos(\pi /2) & -sin(\pi /2)\\ 
	&\\
	sin(\pi /2) & cos(\pi /2) 
	\end{bmatrix}
 = \begin{bmatrix}
	0 & -1\\ 
	1 & 0 
	\end{bmatrix} \addunit{}
\end{equation}

the direction vector can be determined as: 

\begin{equation}	
\vec{F}=\vec{U_L} \cdot R \addunit{}
\label{eq:directionVect}
\end{equation}

The vectors from origin to the wheels can time dependently be described by differential equations:

\begin{equation}
\dot{\vec{X_L}} = r \cdot \omega_L \cdot \vec{F} \addunit{} 
\label{eq:vect2wheelL}
\end{equation}
\begin{equation}
\dot{\vec{X_R}} = r \cdot \omega_R \cdot \vec{F} \addunit{} 
\label{eq:vect2wheelR}
\end{equation}
\startexplain
	\explain{$r$ is the radius of the wheels}{\SI{}{\meter}}
	\explain{$\omega_L$ and $\omega_R$ are the angular velocities of the wheels}{\SI{}{\radian\per\second}}
\stopexplain

Equally, the directional vector $\vec{F}$ and the center point of the vehicle can time dependently be described as:

\begin{equation}
\dot{\vec{F}} 
= R \cdot \frac{\dot{\vec{X_L}} - \dot{\vec{X_R}}}{|\vec{X_L} - \vec{X_R}|}
= R \cdot \frac{r}{|\vec{X_L} - \vec{X_R}|} \cdot \vec{F} \cdot (\omega_L - \omega_R)
 \addunit{} 
\end{equation}
\begin{equation}
\dot{x_c} 
= \frac{\dot{\vec{X_L}} - \dot{\vec{X_R}}}{2} 
= \frac{r}{2} \cdot \vec{F} \cdot (\omega_L - \omega_R) \addunit{} 
\end{equation}

With these equations and the driving calculations, the drive handling can be simulated. Two simulations are performed and are shown in \autoref{fig:MotorSimCurve} and \autoref{fig:MotorSimRoute}.

\begin{figure}[h!]
    \centering
        \includegraphics[width=1\textwidth]{curvemotor}
        \caption{Simulation of soft curve with two points.}
        \label{fig:MotorSimCurve} 
\end{figure}
\begin{figure}[h!]
  \centering
         \includegraphics[width=1\textwidth]{routemotor}
        \caption{Simulation of route following.}
        \label{fig:MotorSimRoute}
\end{figure}

The first simulation, shown in \autoref{fig:MotorSimCurve}, is done with only two points. It is used to illustrate the process which will occur if the robot is suddenly placed off of the line it is supposed to follow. In the simulation the robot is set to drive on the Y-axis until it reaches the point $P=(0,10)$. From here it will drive out to $P=(10,10)$. From this it is concluded that in theory the algorithm can keep the mower on track if supplied with the correct coordinates from the Reach devices. The second simulation, shown in \autoref{fig:MotorSimRoute}, is a simulation of how the algorithm will behave when used on a route set by the previous module. This simulation shows that the algorithm works well, and that it has no problems staying within the perimeter and that it turns well. 
\subsection{Integration of driving}
After validating the model of the vehicle through simulations, the controller is implemented on the Intel Edison board, where the calculations of the angular velocities of the motors will be handled. 
The direction and duty cycle is calculated and sent via a serial communication link to the \gls{fpga}. This information is divided into three bytes as shown in \autoref{fig:byteToFPGA}.

\begin{figure}[h]
\includegraphics[width=\textwidth]{bytedivision}
\caption{Division of information in bytes sent from the \gls{fpga} to the Edison via a \gls{uart} com} 
\label{fig:byteToFPGA}
\end{figure}

To specify if the command is sent to the left or right wheel, an ASCII value corresponding to a capital R or L is sent and an ASCII value corresponding to a lower case f or r specifies whether it is forward or reverse. The last byte states what duty cycle the \gls{pwm} must have in order to get the desired direction. This is sent as an 8 bit value between 0 and 255. Since the motors need a duty cycle of at least \SI{35}{\percent} to start, the second lowest value the integer can obtain is defined as:
\begin{equation}
\frac{255}{100} \cdot 35 = 89.25 \approx 90 
\end{equation}
The transmission of the three bytes in C code is shown in \autoref{cs:serialTrans}.

\includeCode{motor.c}{C}{241}{264}{Serial transmitting of angular velocities for motors.}{cs:serialTrans}{../Kode/clienttest/}


