\subsection{Ultrasonic Sensor}
As stated in \autoref{req:obstacles} the lawn mower must be able to circumnavigate obstacles. In order to locate obstacles the system will use ultrasonic sensors to scan for obstacles and measure the distance to these. As explained in \autoref{subsec:UltraSound} an ultrasonic sensor emits sound waves and measures the time from transmission to receiving the wave. This wave has a specified speed in air and when the controller knows the time from transmitted to received, the controller can calculate the distance by using the following formula: 

	\begin{equation}
	\text{distance} = \frac{1}{2}\cdot  v_{sound} \cdot t \addunit{m}
	\end{equation}
	
	\startexplain
		\explain{$t$ is the time}{\SI{}{\second}}
		\explain{$v_{sound}$ is the speed of sound}{\SI{}{\meter\per\second}}
	\stopexplain

The variable, $v_{sound}$ is sensitive to more than temperature but the formula is an approximation because temperature has the greatest influence, and therefore all other variables are considered as non significant \citep{SoundSpeed}. The formula is: 

	\begin{equation}
		v_{sound}=\sqrt{\frac{\gamma \cdot R \cdot T}{M}} \addunit{m/s}
	\end{equation}

	\startexplain
		\explain{$\gamma$ is the adiabatic constant}{1}
		\explain{$R$  is the gas-constant}{\si{\kelvin\joule\per\mole}}
		\explain{$T$ is the absolute temperature}{\si{\kelvin}}
		\explain{$M$ is the molecular mass of gas}{\si{\kilogram\per\mol}}
	\stopexplain

The adiabatic constant in air is approximated to 1.4 and the molecular mass of gas has an average molecular mass for dry air of 0.02895 \SI{}{\kilogram\per\mol} and the gas constant is 8.314 \SI{}{\kelvin\joule\per\mol}, this leads to the following formula \citep{SoundSpeed}:

	\begin{equation}
		v_{sound}= 20.05\cdot \sqrt T\addunit{m/s}
	\end{equation}

Since ultrasonic waves have travel in a specific frequency band, and because it is important to avoid any interference, a filter must be constructed, to ensure that signals with other frequencies than that of the ultrasonic waves are attenuated. This is done, so the only interference possible is from signals of the same frequency. For an ideal realization of this, a bandpass filter where only waves of the wanted frequency gets through is designed in \autoref{app:filterdesign}, where the bandpass frequency is chosen to \SI{40}{\kilo\hertz}.

When acquiring components such as an ultrasonic sensor, a filter is often implemented from the manufacturer, removing the need of constructing an external filter. One such component is the ultrasonic ranging module, HC-SR04 shown in \autoref{fig:HC-SR04}.

\begin{figure}[h!]%
	\centering
	\includegraphics[width=\columnwidth]{HC-SR04}
	\caption{Ultrasonic ranging module HC-SR04.}
	\label{fig:HC-SR04}%
\end{figure}

The HC-SR04 is an ultrasonic ranging module with a \SI{2}{\centi\meter} to \SI{400}{\centi\meter} non-contact measuring function. The module consists of an ultrasonic transmitter, receiver, and a control circuit. The principle behind the sensor is that the transmitter receives a high logic signal for \SI{10}{\micro\second} and then it sends eight pulses of ultrasonic waves with a frequency of \SI{40}{\kilo\hertz}, the receiver then detects if there is a pulse signal returned. The module works best within an angle of \SI{30}{\degree} and  the resolution of the output is \SI{1}{\centi\meter} \citep{HC-SR04}.

\begin{figure}[h!]%
	\centering
	\includegraphics[width=\columnwidth]{bilmedvinger}
	\caption{Sketch of vehicle with five ultrasonic module mounted.}
	\label{fig:bilmedvinger}
\end{figure}

As seen in \autoref{fig:bilmedvinger} the vehicle will be mounted with five ultrasonic modules. There are three modules mounted on the front of the vehicle, this is done to assure that the entire front of the vehicle is covered, so there will not be any frontal collisions with any bigger obstacles. The modules on each side of the vehicle have the purpose of giving the system a possibility to track an obstacle on the side of the vehicle as it is passed. This will also assure that the obstacle detection system will be able to go around the obstacle with a fixed distance to the obstacle, and thus minimizing the alteration of the planned route.

Since five sensors are used in the design there are a few problems timing wise, since the sensors on the front of the lawn mower cannot run at the same time because of possible interference. This would make the system very slow, since each sensor needs a minimum of \SI{70}{\micro\second} to finish the measuring \citep{HC-SR04}. The two sensor modules on the side of the lawn mower can run continuously, but since the three sensors on the front cannot it will take \SI{210}{\micro\second}, before all sensors have sampled once. The ultrasonic obstacle detection will be implemented on an \gls{fpga} using \gls{vhdl}.

The \gls{vhdl} software for the ultrasonic ranging system is built up of four \gls{vhdl} modules. The different modules are: a generic counter, a trigger generator, a distance calculation, and a top module to combine the three other modules. A block diagram for the functionality of the ultrasonic ranging system is shown in \autoref{fig:ultra_block}.

\begin{figure}[h!]%
	\centering
	\includegraphics[width=\columnwidth]{Ultrasonic_block}
	\caption{Block diagram of the modules in the ultrasonic ranging system}
	\label{fig:ultra_block}
\end{figure}

The first module of the Range sensor system is the counter, and the behaviour of the counter is shown in \autoref{cs:vhdlcount}.

\includeCode{Counter.vhd}{VHDL}{33}{58}{The generic counter module}{cs:vhdlcount}{../Kode/VHDL-Ultrasonic/}

The counter is designed as a generic counter, and the idea behind this is that it can easily be used in all of the other modules. The counter is designed so it counts from $n-1$ to $0$ and the individual modules define the value of $n$.

The next module of the system is the trigger generator. The behaviour of this module can be seen in \autoref{cs:vhdltriggen}.

\includeCode{Trigger_generator.vhd}{VHDL}{48}{66}{The trigger generator module}{cs:vhdltriggen}{../Kode/VHDL-Ultrasonic/}

The trigger generator works in the way, that it specifies the size of the genereic counter. It specifies two constants, where the first is \SI{250}{\milli\second}, and the other constant is \SI{250}{\milli\second} plus an additional \SI{100}{\micro\second}. The trigger is then set to high when the counter reaches the first constant and then set low again, when the counter reaches the second constant. If the second constant is equal to the counter output the counter is reset, which means it is set to low, because the reset of the generic counter is active low. This makes the trigger generate a pulse roughly four times every second, and it also makes sure the trigger sends out a burst of ultrasonic waves that last \SI{100}{\micro\second} as specified in the data sheet \citep{HC-SR04}.


\includeCode{Distance_calculation.vhd}{VHDL}{55}{73}{The distance calculation module}{cs:vhdlciscalc}{../Kode/VHDL-Ultrasonic/}

The distance calculation works in the way, that it uses the two variables, result and multiplier. Where the result is an integer, and the multiplier is where the calculation is done and temporarily stored. The datasheet states, that to get the distance in centimeters, the time must be in \si{\micro\second} and divided by 58. Since the \gls{fpga} clock is \SI{32}{\mega\hertz} the time of the pulse will be in \si{\nano\second}, and this must be converted to \si{\micro\second} by multiplying the pulse with the clock period and dividing by 1000. Division is however somewhat tricky in \gls{vhdl}, so instead the pulsewidth is multiplied by three and shifted to the 13th position to the right. This gives a value of 11 bits, and it is then checked if the sensor reaches to its highest value, if not the result is stored in a nine bit distance signal after further dividing by three, which is done by shifting two times to the right. This gives us an output distance signal of nine bits, which will be the distance in centimeters. 

Due to time constraints of the project, it is decided not to implement  ultra sonic detection system in the prototype despite the modules already being designed. This is mainly because the software that will enable the mower to change direction based on the sensor input is not developed. 