\section{Gearing}

The gearbox from motor to wheel has the ratio 75:1. This ratio is the combined ratio from the motor, which is 50:1 and from the vehicle which have a gearing ratio to 1.5:1. The free body diagram is as seen on \autoref{fig:free-body}.
From the free body diagram, differential equations for the motors can be derived and Laplace transformed in order to make a block diagram for the motor. The derivation of these equations can be seen in \autoref{app:freebody_motor}. The calculations end up in \autoref{Eq:done}.

\begin{figure}[htb] 
\begin{center}

\begin{picture}(0,0)%
\includegraphics[width=.6\textwidth]{gear}%
\end{picture}%
\setlength{\unitlength}{2368sp}% Unscaled unitlength: 3947sp
%
\begingroup\makeatletter\ifx\SetFigFont\undefined%
\gdef\SetFigFont#1#2#3#4#5{%
  \reset@font\fontsize{#1}{#2pt}%
  \fontfamily{#3}\fontseries{#4}\fontshape{#5}%
  \selectfont}%
\fi\endgroup%
\begin{picture}(6698,5569)(811,-5382)
\put(6361,  4){$f_b$}%
\put(3581,-2376){$f_g$}%
\put(5606,-2716){$J_p$}%
\put(5921,-571){$r_3$}%
\put(5326,-1486){$r_2$}%
\put(7276,-2761){$B_p\omega_p$}%
\put(2401,-2461){$J_m$}%
\put(4776,-1261){$f_g$}%
\put(4276,-2911){$\omega_p$}%
\put(746,-1561){$K_i$}%
\put(2551,-211){$\omega_m$}%
\put(2701,-1486){$r_1$}%
\put(2940,-4394){$f_b$}%
\put(2890,-5304){$B_b\dot{x}$}%
\put(4120,-4534){$M_b$}%
\put(2101,-3161){$B_m\omega_m$}%
\put(3650,-3594){$x$}%
\end{picture}%
\caption{This picture shows the free-body diagram from motor to wheel.} \label{fig:free-body} 
\end{center}
  \end{figure}


\vspace{-1mm}

\begin{equation}
\left(J_m+J_p\left(\frac{r_1^2}{r_2^2}\right)+M_b r_3^2\left(\frac{r_1^2}{r_2^2}\right)\right)s\omega_m(s)=KI-\left(B_m+B_p\left(\frac{r_1^2}{r_2^2}\right)+B_b r_3^2\left(\frac{r_1^2}{r_2^2}\right)\right)\omega_m(s)
\label{Eq:done}
\end{equation}
\startexplain
\explain{$J_m$  is the total moment of inertia of the motor}{\SI{}{\newton\meter}}
\explain{$J_p$  is the total moment of inertia on the gear}{\SI{}{\newton\meter}}
\explain{$M_b$  is the total moment of inertia of the lawn mower mass}{\SI{}{\newton}}
\explain{$K_i$  is the input current times motor constant}{\SI{}{\volt\second\per\radian}}
		\explain{$\omega_m$  is the motor angular velocity }{\SI{}{\radian\per\second}}
		\explain{$\omega_p$  is the pulley angular velocity }{\SI{}{\radian\per\second}}
		\explain{$r_1$  is the gear radius on motor }{\SI{}{\meter}}
		\explain{$r_2$  is the gear radius on pulley }{\SI{}{\meter}}
		\explain{$r_3$  is the pulley wheel radius }{\SI{}{\meter}}
		\explain{$B_m$  is the friction coefficient on motor shaft}{\SI{}{\newton\meter\second}}
		\explain{$B_p$  is the friction coefficient on gear}{\SI{}{\newton\meter\second}}
		\explain{$B_b$  is the friction coefficient on the belt}{\SI{}{\newton\second\per\meter}}
	\stopexplain 
This equation can be divided into the following three equations:

\begin{subequations}
\begin{equation}
J_{tot}=J_m+J_p\frac{r_1^2}{r_2^2}+M_br_3^2\frac{r_1^2}{r_2^2}
\end{equation}

\begin{equation}
B_{tot}=B_m+B_p\frac{r_1^2}{r_2^2}+B_br_3^2\frac{r_1^2}{r_2^2}
\end{equation}

\begin{equation}
J_{tot}s\omega_m(s)=KI-B_{tot}\omega_m(s)
\end{equation}
\end{subequations}

From these three equations a block diagram is created, as shown in \autoref{fig:block_diagramdrive}.

\begin{figure}[htb] 
	\begin{center} 
		\begin{picture}(0,0)%
		\includegraphics{gearblock.pdf}%
		\end{picture}%
		\setlength{\unitlength}{4144sp}%
		%
		\begingroup\makeatletter\ifx\SetFigFont\undefined%
		\gdef\SetFigFont#1#2#3#4#5{%
			\reset@font\fontsize{#1}{#2pt}%
			\fontfamily{#3}\fontseries{#4}\fontshape{#5}%
			\selectfont}%
			\fi\endgroup%
			\begin{picture}(6297,1284)(836,-397)
			\put(4184,563){$\frac{r_1}{r_2}$}%
			\put(5287,554){$\frac{1}{s}$}%
			\put(6358,554){$r_3$}%
			\put(4573,697){$\omega_p(s)$}%
			\put(2926,-241){$B_{tot}$}%
			\put(6706,704){$x(s)$}%
			\put(5626,704){$\theta_p(s)$}%
			\put(1531,569){$K$}%
			\put(1036,704){$I$}%
			\put(2566,344){$-$}%
			\put(2116,434){$+$}%
			\put(1891,704){$\tau_m(s)$}%
			\put(2986,556){$\frac{1}{J_{tot}}$}%
			\put(3483,697){$\omega_m(s)$}%
			\end{picture}%
			\caption{This figure shows the transfer function from motor to wheel.} \label{fig:block_diagramdrive} 
			\end{center}
			\end{figure}



	