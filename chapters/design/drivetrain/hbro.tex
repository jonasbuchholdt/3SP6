\section{H-bridge}\label{sec:hbro}
In order to drive the motors an H-bridge is needed. This is mainly because the current delivered by the \gls{fpga} is too low and with the implementation of an H-bridge, the current can be drawn directly from the power supply. An H-bridge is a construction of switches (typically in the form of transistors) that resembles the shape of an H. A standard H-bridge setup is depicted in \autoref{fig:h-bridge}. 

\begin{figure}[htb]
\centering
\includegraphics[width=0.5\textwidth]{hbro-design}
\caption{Standard H-bridge circuit. Notice the diodes which prevents back EMF current \citep{hbridge}.}
\label{fig:h-bridge}
\end{figure}

Depending on which switches is opened, the current can run through the load from both directions. 

%\begin{figure}[htb]
%    \centering
%    \begin{subfigure}[b]{0.45\textwidth}
%        \includegraphics[width=1\textwidth]{simple_bridge_bw}
%        \caption{Current flow when Q1 and Q4 are opened}
%        \label{fig:currentbw}
%    \end{subfigure}
%    \begin{subfigure}[b]{0.45\textwidth}
%         \includegraphics[width=1\textwidth]{simple_bridge_fw}
%        \caption{Current flow when Q2 and Q3 are opened}
%        \label{fig:currentfw}
%    \end{subfigure}
%    \caption{Illustration of points inside and outside area connected to all corners, to form four triangles.}\label{fig:hbridge-direction}
%\end{figure}

However controlling the transistors must be done with very precise timing. This is because the switching from open to closed is not instantaneous. If Q2 opens before Q1 is closed the power supply will short-circuit. In order to control which transistors open and when this happens a controller is used. This controller dictates whether the transistors are opened or not, and takes care of the timing. The H-bridge controller chosen for this project is a DRV8704 chip. This chip can control two H-bridges at once and therefore only one chip is needed to control a H-bridge for each of the two motors on the mower. 

The controller is implemented on a PCB and mounted on the chassis.

%\begin{figure}[htb]
%\centering
%\includegraphics[width=0.5\textwidth]{hbro-designprint}
%\missingfigure{Print layout af hbro controlleren}
%\caption{PCB layout for the H-bridge controller design}
%\label{fig:h-bridgeprint}
%\end{figure}