\section{Motor}
The chosen motor is a Krick MAXgear 50:1, with a maximum voltage rating of \SI{12}{\volt} and a gearing ratio of 50:1. The motor has a torque of \SI{295}{\newton\centi\meter} and maximum revolution of \SI{316}{U/min} \citep{Krick}. The circuit equivalent of the motor can be seen in \autoref{fig:motor_circuit_equivalent}.

\begin{figure}[h!]
\centering
\begin{circuitikz}\draw (0,0) node[left] {$v_+$}
to[R=$R_a$, i>^=$i_a$, o-]  (3,0) 
to[L=$L_a$] (5,0)
to[V=$K \omega $] (5,-3)
to[short, -o] (0,-3) node[left] {$v_-$}
;\end{circuitikz}
\caption{This picture shows a standard circuit equivalent for a DC motor.}
\label{fig:motor_circuit_equivalent}
\end{figure}

To enable the functionality of turning around a center point on the vehicle, two motors are used. To make sure that these motors are similar, measurements are made on each motor and compared to the other. The measurements are done to find and compare the following parameters for each motor: The armature resistance $R_a$, the armature inductance $L_a$, the back EMF $K_\omega$ and the rotational damping $B$. These parameters are also found in order to be able to simulate and develop a controller for the motors. The test journal for the motor measurements is seen in \autoref{app:mot_meas}, and the results for each motor is seen in \autoref{tab:mot_comp}. 
\begin{table}[htb]
\centering
\caption{Constants of the two motors in the drive train.}
\label{tab:mot_comp}
\begin{tabular}{ c r r r r }
\textit{Motor} & \textit{Resistance} & \textit{Inductance} & \textit{Motor constant} & \textit{Friction} \\
 & [\si{\ohm}] & [\si{\milli\henry}] & [\si{\milli\volt\,\radian\,\second$^{-1}$}] & [\si{\micro\newton\,\meter\,\second\,\radian$^{-1}$}] \\
\toprule 
\rowcolor{lightGrey}
\textit{1} & 0.83 & 0.77 & 7.0 & 1.09 \\
\textit{2} & 0.73 & 0.69 & 6.4 & 0.77 

\end{tabular}
\end{table}

The two motors are not completely identical, but the difference between the two motors on each of the four parameters is very small, so the two motors can be approximated as identical when modelling these.
