\section{PWM}\label{subsec:motor_pwm}
In this section, the \gls{pwm} signal will be designed. The design is made for the worst case scenario, where the rotor is fixed, meaning $\omega = 0$, and with a duty cycle of $d=0.5$. The first step is choosing the percent of ripple tolerated in the current. If the ripple is too high, the motor will generate an unnecessary amount of heat, so it is important to choose a small peak to peak percent ripple. A drawback when having to ensure a very low ripple percentage is that it requires a very fast switching frequency. To ensure that the ripple is not too high and the switching frequency is not too high either, a ripple of $5 \%$ is chosen.

The formula to calculate the switching frequency is as following:
\begin{equation}
	f_{sw}\geq -\frac{1}{2\ln\left(1-\frac{p}{100}\right)}\frac{R_a}{L_a}
\end{equation}
\startexplain
	\explain{$p$ is the ripple percentage}{1}
\stopexplain
The switching frequency is calculated for motor one and two using the data from \autoref{tab:mot_comp}. The highest of the two is selected as the switching frequency, $f_{sw}$.

\paragraph{Motor one:}
\begin{subequations}
	\begin{equation}
	f_{sw} \geq  -\frac{1}{2\ln\left(1-\frac{5}{100}\right)}\frac{\SI{0.84}{\ohm}}{\SI{0.77}{\milli\henry}}
	\end{equation}
	
	\begin{equation}
		f_{sw} \geq  \SI{10\,634}{\hertz}
	\end{equation}
	\label{Eq:fsw}
\end{subequations}

\vspace{-10mm}
\paragraph{Motor two:}
	\begin{subequations}
		\begin{equation}
			f_{sw} \geq  -\frac{1}{2\ln\left(1-\frac{5}{100}\right)}\frac{\SI{0.73}{\ohm}}{\SI{0.69}{\milli\henry}} 
		\end{equation}
		
		\begin{equation}
			f_{sw} \geq  \SI{10\,312}{\hertz}
		\end{equation}
	\end{subequations}

From \autoref{Eq:fsw} the switching frequency needs to be at least  \SI{10\,634}{\hertz}. To ensure this, when calculating with a $5 \%$ ripple the following formula is used:

\begin{subequations}
	\begin{equation}
		f_{sw} \geq 10\,634+\frac{10\,634\cdot 5}{100}
	\end{equation}
	
	\begin{equation}
		f_{sw} \geq  \SI{11\,166}{\hertz} 
	\end{equation}
	
\end{subequations}
