\section{Digital Signal Processor Choice}

The chosen component is the \gls{dsp} for numerous reasons that are going to be presented in the following subsections.

\subsection{Advantage of the \gls{dsp} over the FPGA}

There are many advantages to choose a \gls{dsp} over an FPGA in this project. 

The first one is the simplicity of designing with a \gls{dsp} due to the availability of all the simple functionalities in a micro-controller \citep{eetimes}. \\

\gls{dsp} is designed to perform signal processing task unlike the FPGA which is purpose of creation was not related to signal processing but to program circuits \citep{eetimes}. \\

According to different sources, a \gls{dsp} is more suitable in terms of cost/performance for an MMAC lower than 3000, which means that it is better for low demanding applications \citep{eetimes}. \\

Additionally, FPGAs lack integrated DAC and ADC \citep{eetimes}. \\

The \gls{dsp}s are more power efficient than the FPGAs which can be important if the effect is to be used on the move on battery power \citep{rtcmag}. \\

In terms of size, the \gls{dsp} fills less space than the FPGA. A scheme regrouping the power and space considerations is shown in figure ??.  A smaller device is always preferable for a musician \citep{offchip}. \\

A \gls{dsp} can be optimal if there are many operations that are repeated. From the analysis part section??, it is known that only 6 types of blocks are going to be used in order to create all the stated effects \citep{eetimes} \citep{hunteng}. \\

\subsection{Advantage of the \gls{dsp} over the Raspberry Pi}

\gls{dsp}s are much faster for integer and floating mathematical calculations.  \\

A Micro-controller has many different features that are not going to be used in the project. \\

A \gls{dsp} is faster than a micro-controller whereas a micro-controller has more features but they are not going to be used in the project. \\

A micro-controller is usually used for control applications and not signal processing even if it can work.








