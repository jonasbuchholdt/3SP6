\section{Description of the chosen DSP}


The chosen \gls{dsp} is the Texas Instruments TMS320C5515 Fixed-Point Digital Signal Processor.
The \gls{dsp} can perform at four different clock rates, dependent on the supply voltage. It can run \SI{60}{\mega\hertz} or \SI{75}{\mega\hertz}, if the supply voltage is \SI{1.05}{\volt}, giving cycle times of \SI{16.67}{\nano\second} or \SI{13.3}{\nano\second}. If the supply voltage is\SI{1.3}{\volt} the \gls{dsp} can run a \SI{100}{\mega\hertz}- or \SI{120}{\mega\hertz} clock, giving cycle times of \SI{10}{\nano\second} or \SI{8.33}{\nano\second}. The \gls{dsp} can to one/two instructions per cycle. An example of when the \gls{dsp} can do two instructions per cycle, is when a multiplication and an addition is done. The \gls{dsp} has two \gls{mac}'s, making it able to do one 17-bit times 17-bit multiplication and one 32-bit addition in one cycle. This means the \gls{dsp} can do potentially $2 \cdot 120 \cdot 10^6 = 240$ \gls{mips}. \todo[inline]{ask sofus if it is correct}. The \gls{dsp} has a 8 to 1024 point 16-bit \gls{fft} hardware accelerator, and supports \gls{uart}, \gls{spi}, \gls{i2c} and \gls{i2s} communication protocols.
The \gls{dsp} has \SI{448}{\kilo\byte} on-chip memory divided in \SI{320}{\kilo\byte} \gls{ram} and \SI{128}{\kilo\byte} \gls{rom}. The \gls{ram} is divided in \SI{256}{\kilo\byte} \gls{saram} and \SI{64}{\kilo\byte} \gls{daram}, while all the \gls{rom} is \gls{sarom}.
%
%\subsection{Power}
%
%The DSP processor gets its power from the \gls{usb} connection.
%
%\subsection{Different Ports}
%
%The C5515 features multiple ports and connectors. It has a micro SD connector, an expansion edge connector as well as a two \gls{usb}s. It also has an audio IN and two Bluetooth interfaces. \\
%
%Each USB connector has a different role, one is used for software purposes and the other one to connect to the personal computer. 
%
%\subsection{LCD Screen}
%
%The \gls{dsp} processor has also an LCD screen that can print different characters by using a string of 14 pins. 
%
%\subsection{Interface with the User}
%
%The processor contains two switch buttons that can be integrated in the software to create an action. 
%
%\subsection{Expansion}
%
%The expansion connectors can be used to redirect the signals to other user interfaces that can be attached to some of the pins on the DSP. The exact number of the pins is in the data sheet. 
%
%\subsection{Monitoring and Testing}
%
%Several "Test points" are available in order to see the signals' behavior moving in the \gls{dsp} processor.
%
%\subsection{Performance}
%
%The \gls{dsp} can run on different speeds depending on how much the \gls{cpu} is powered. The clock rate can be between 60 and 75 MHz if powered with 1.05 Volts and between 100 and 120 MHz if powered with 1.3 Volts. \\
%
%\subsection{Memory}
%
%The chosen DSP has 320 \gls{kb} of \gls{ram} and 128 \gls{kb} of \gls{rom}. \\
%The \gls{ram} consists of a 64 \gls{kb} of dual-access \gls{ram}, 256 \gls{kb} single-access \gls{ram} which clearly gives 320 \gls{kb}. \\
%Dual-access and single access \gls{ram}s presented above can be accessed using an internal program, data or \gls{dma} buses. \\
%Dual-access and single access \gls{ram}s are divided into blocks that can be accessed using byte address ranges. Ranges are defined for each block but all blocks of a same memory type can be regrouped into one address range. \\
%\gls{dma} address range for the block 0 of the dual-access \gls{ram} is 0001 0000h – 0001 1FFFh for instance. \\
%All of the addresses are presented in the data sheet tables 2-2, 2-3.
%The single-access \gls{rom} is also divided into blocks and has an address range. The usable range can be changed using a software. \\
%
%The external memory interface permits a connection to other memories from other devices than the \gls{dsp}. 