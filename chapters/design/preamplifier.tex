\section{Pre-amplifier}
In order to transfer as much signal from the guitar as possible, and avoid some significant voltage division between the guitar and the \gls{dsp} in some frequency area, a \gls{preamp} will be designed. The output impedance of the guitar is measured an the result is shown in \autoref{app:output_impedance}. This shows that the output impedance for a guitar is not the same from \SI{10}{\hertz} to \SI{22}{\kilo\hertz} and the output impedance is generally very high. Therefore the \gls{preamp} needs to have an very high input impedance to avoid significant voltage division in the frequency area from \SI{10}{\kilo\hertz} to \SI{22}{\kilo\hertz} compared to the frequency area from \SI{10}{\hertz} to \SI{10}{\kilo\hertz}. The input impedance shall therefore be much higher than \SI{73.06}{\kilo\hertz} which is the highest measured output impedance on the guitar, to allow as much signal transfer as possible.

\subsection{Component chose}

As seen in the \autoref{app:guitar_max_amplitude} that the maximum output signal peak amplitude from the guitar is about $\SI{1}{\volt}_{peak}$ and the standard in ?? says that the signal from a line source shall have an voltage \gls{rms} value of $\SI{1.768}{\volt}_{\gls{rms}}$ which have a peak amplitude of $\SI{2.5}{\volt}_{peak}$. This means that the \gls{preamp} shall have a gain of approximate \SI{8}{\decibel}. Since the amplification only shall be approximate \SI{8}{\decibel} and the input impedance shall be very high, an \gls{opamp} will fit very well. An \gls{opamp} often have an input impedance of more than \SI{1}{\mega\ohm} and have a gain of about \SI{100}{\decibel}, and after the feedback circuit is implemented on the \gls{opamp} the input impedance is even higher and the output impedance is even smaller.

\subsection{The amplifier}
The \gls{opamp} can ether be working in non inverting or inverting \gls{opamp} configuration, but since the \gls{preamp} shall have en gain of \SI{8}{\decibel} and a high input impedance, the non inverting \gls{opamp} configuration will fit very well. The non inverting \gls{opamp} configuration have at least a gain of \SI{0}{\decibel} and it is easy to design a high input impedance. The amplification is given by \autoref{eq:amplification}

\begin{equation}\label{eq:amplification}
        V_{out} =V_{in} \cdot (1+\frac{R_2}{R_1})
        \addunit{\si{\volt}}
    \end{equation}

    \startexplain
        \explain{$V_{out}$ is the output voltage}{\si{\volt}}
        \explain{$V_{in}$  is the input voltage}{\si{\volt}}
        \explain{$R_1$ is a resistor in the feedback circuit}{\si{\ohm}}
        \explain{$R_2$ is a resistor in the feedback circuit}{\si{\ohm}}
    \stopexplain
    
    