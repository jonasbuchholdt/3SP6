\section{Pre-amplifier}
In order to transfer as much signal from the guitar as possible, and avoid some significant voltage division between the guitar and the \gls{dsp} in some frequency area, a \gls{preamp} will be designed. The output impedance of the guitar is measured an the result is shown in \autoref{app:output_impedance}. This shows that the output impedance for a guitar is not the same from \SI{10}{\hertz} to \SI{22}{\kilo\hertz} and the output impedance is generally very high. Therefore the \gls{preamp} needs to have an very high input impedance to avoid significant voltage division in the frequency area from \SI{10}{\kilo\hertz} to \SI{22}{\kilo\hertz} compared to the frequency area from \SI{10}{\hertz} to \SI{10}{\kilo\hertz}. The input impedance shall therefore be much higher than \SI{73.06}{\kilo\hertz} which is the highest measured output impedance on the guitar, to allow as much signal transfer as possible.

\subsection{Component chose}

As seen in the \autoref{app:guitar_max_amplitude} that the maximum output signal peak amplitude from the guitar is about $\SI{1}{\volt}_{peak}$ and the 
\gls{dsp} have a capasitivity of maximum $\SI{2222}{\volt}_{peak}$. To make a general \gls{preamp} that works with more guitar than only the measured guitar, the \gls{preamp} will be designed with a volume control. This means that the \gls{preamp} shall have a gain of approximate \SI{99999}{\decibel} and the input impedance shall be over a decade higher than the guitar output impedance.
	An \gls{opamp} often have an input impedance of more than \SI{1}{\mega\ohm} and have a gain of about \SI{100}{\decibel}, and after the feedback circuit is implemented on the \gls{opamp} the input impedance is even higher and the output impedance is even lower. A simple diagram over the \gls{preamp} is shown in \autoref{fig:simple_preamp} 

\begin{figure}[h!]
\centering
\begin{circuitikz}\draw (0,0)
(0,0)to[short, o-]node[left,above]{$Input$}(1,0)
to[amp, t=$G1$]  (3,0)
to[short] (3,0)
to[pR=$ $](3,-2)
to[short](3,-2)node[ground]{}(3.5,-1)
to[short] (4,-1)
to[short] (4,0)
to[amp, t=$G2$]  (6,0)
(6,0)to[short, -o]node[left,above]{$Output$}(7,0)
%to[short] (0,0)
;\end{circuitikz}
\caption{A simple block diagram of a \gls{preamp}}
\label{fig:simple_preamp}
\end{figure}


\subsection{The amplifier $G1$}
The \gls{opamp} part $G1$ in \autoref{fig:simple_preamp} can ether be working in non inverting or inverting \gls{opamp} configuration, but since the \gls{preamp} shall have a gain of \SI{99999}{\decibel} and a high input impedance, the non inverting \gls{opamp} configuration will fit very well, because it is a Voltage-seriel feedback configuration. The seriel feedback configuration means that the input impedance is higher than a parallel feedback, because the feedback circuit is in seriel with the input imperdance of the \gls{opamp} The schematics of a non inverting \gls{opamp} is shown in the following schematic \autoref{fig:preamp_opamp}

\begin{figure}[h!]
\centering
\begin{circuitikz}\draw (0,0)
node[op amp,yscale=-1] (opamp) {} 
(-3,-3.5)node[ground]{}(-3,-3.5)
to[R=$R_{Bias}$] (-3,0.5)
to[short](opamp.+) 
(-5,0.5)to[short, o-]node[left,above]{$Input$} (-3,0.5)
(opamp.-) 
(opamp.out) 
to[short] (2,0)
to[short] (2,-1.5)
to[R=$R_F$] (-1.5,-1.5)
to[short] (-1.5,-0.5)
to[short] (-1,-0.5)
(-1.5,-1.5)to[R=$R_1$] (-1.5,-3.5)node[ground]{}(-2,-3.5)
(2,-1.5)to[pR, l_=$R_V$] (2,-3.5)node[ground]{}(2,-3.5)
(2.5,-2.5)to[short, -o]node[left,above]{$Output$} (4,-2.5)
;\end{circuitikz}
\caption{The schematic over $G1$ \autoref{fig:simple_preamp}}
\label{fig:preamp_opamp}
\end{figure}

The output is looking into another \gls{opamp} with over \SI{1}{\mega\ohm}, so the calculation is without the another \gls{opamp}, because it look like an open circuit. The equivalent schematic of a non inverting \gls{opamp} is as following \autoref{fig:preamp_opamp_equa}

\begin{figure}[h!]
\centering
\begin{circuitikz}\draw (0,0)
to[R=$R_{Bias}$] (0,-6)node[ground]{}(0,-6.5)
(0,0)to[short] (2,0)
to[R=$Z_{i}$] (2,-2)
to[R=$Z_{i\beta}$] (2,-4)
to[V=$\beta \cdot V_o$] (2,-6)node[ground]{}(2,-6.5)
(0,0)to[short, -o]node[left,above]{$Z_{Input}$} (-3,0)
;\end{circuitikz}
\caption{The equalent schematic over $G1$ \autoref{fig:simple_preamp}}
\label{fig:preamp_opamp_equa}
\end{figure}

\newpage

Calculating for $Z_{i\beta}$ is as following \autoref{eq:preamp_ib}

\begin{equation}\label{eq:preamp_ib}
        Z_{i\beta} = R_F\parallel R_1
    \end{equation}

    \startexplain
        \explain{$R_1$ is a resistor in the feedback circuit}{\si{\ohm}}
        \explain{$R_F$ is a resistor in the feedback circuit}{\si{\ohm}}
    \stopexplain

The resulting input impedance will be as following \autoref{eq:preamp_result} because an voltage supply is a shut circuit when calculating impedance.

\begin{equation}\label{eq:preamp_result}
        Z_{Input} = R_{Bias}\parallel Z_i + Z_{i\beta}
    \end{equation}

    \startexplain
        \explain{$Z_{Input}$ Resulting input impedance}{\si{\ohm}}
        \explain{$R_{Bias}$ An bias resistor}{\si{\ohm}}
        \explain{$Z_i$ The input impedance of the \gls{opamp}}{\si{\ohm}}
        \explain{$Z_{i\beta}$ the impedance of the feedback circuit}{\si{\ohm}}
    \stopexplain

The feedback circuit will increase the $Z_{Input}$, but since the input impedance of a \gls{opamp} is over \SI{1}{\mega\ohm}, and the feedback circuit will increase the output impedance, the feedback resistors shall be chosen small enough to keep the output impedance low. The following circuit \autoref{fig:preamp_opamp_equa_out} shows the output impedance equerlent circuit of the \gls{opamp}

\begin{figure}[h!]
\centering
\begin{circuitikz}\draw (0,0)
to[V=$A \cdot V_i$] (0,-3)
(0,0)to[R=$Z_o$](2,0)
to[R=$Z_{o\beta}$](2,-3)
(2,0)to[short](4,0)
(4,0)to[pR, l_=$R_V$](4,-3)
(0,-3)to[short, -o](6,-3)
(4.5,-1.5)to[short, -o](6,-1.5)
;\end{circuitikz}
\caption{The equalent schematic over $G1$ \autoref{fig:simple_preamp}}
\label{fig:preamp_opamp_equa_out}
\end{figure}

Calculating for $Z_{o\beta}$ is as following \autoref{eq:preamp_ib_out}

\begin{equation}\label{eq:preamp_ib_out}
        Z_{o\beta} = R_F+R_1
    \end{equation}

    \startexplain
        \explain{$R_1$ is a resistor in the feedback circuit}{\si{\ohm}}
        \explain{$R_F$ is a resistor in the feedback circuit}{\si{\ohm}}
    \stopexplain

Because the output of $R_V$ is loaded by an \gls{opamp}, the $R_F$ is chosen to \SI{5}{\kilo\ohm} to keep the output impedance at the same size as the $R_V$ which is chosen to be \SI{20}{\kilo\ohm}


The amplification is given by \autoref{eq:preamp_amplification}

\begin{equation}\label{eq:preamp_amplification}
        V_{out} =V_{in} \cdot (1+\frac{R_F}{R_1})
        \addunit{\si{\volt}}
    \end{equation}

    \startexplain
        \explain{$V_{out}$ is the output voltage}{\si{\volt}}
        \explain{$V_{in}$  is the input voltage}{\si{\volt}}
        \explain{$R_1$ is a resistor in the feedback circuit}{\si{\ohm}}
        \explain{$R_F$ is a resistor in the feedback circuit}{\si{\ohm}}
    \stopexplain
   