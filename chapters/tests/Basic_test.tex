\section{Test of the Basic requirement}
This section discusses whether the basic requirements stated in section \ref{sec:basic_req} are fulfilled or not. 
In all cases, the requirement \ref{req:dsp} is fulfilled since all the effects are coded in Assembly.


\subsection{Test of requirement \autoref{req:speed}}

In the Appendix \ref{app:effect_run_time}, the runtime of the reverb effect, which gives the elapsed time from the moment a new sample arrives until it is loaded for output, has been tested. \\
Figure \ref{fig:reverb_time_test} shows that only 2.5 microseconds are needed to run the effect. \\
Figure \ref{fig:flanger_time_test} shows that the flanger runtime is 2.3 microseconds. \\
The results show that the runtimes of the effects are clearly faster than the sampling time of the \gls{dsp} which is 23 microseconds. \\
It can be concluded that the effects are real-time, faster than the sampling frequency thus the requirement \ref{req:speed} is fulfilled. \\

\subsection{Test of requirement \autoref{req:SNR}}
Requirement \autoref{req:SNR} states that the \gls{snr} of the final system must be larger than the \gls{snr} in the guitar. In \autoref{app:preamp_frequency_response} the \gls{snr} in the guitar is tested to be \SI{70.06}{\decibel}. In \autoref{sec:dsp_description} it is seen that the \gls{adc} on the TMS320C5515 eZdsp development board has a \gls{snr} of \SI{93}{\decibel}. This states that the requirement could be theoretically approved, but it has not been measured if the implemented effects have had an negative effect on the \gls{snr}. Furthermore the \gls{snr} of the \gls{preamp} has not been measured, which is also a part of the final system. Thus the requirement can not be approved. 

\subsection{Test of requirement \autoref{req:power}}
Requirement \autoref{req:power} states that the maximum output of any of the effects must not be larger than what the \gls{dac} can represent. The \gls{dac} on the TMS320C5515 eZdsp development board has a full scale output voltage of 0.5 V$_{RMS}$ \citep{TLV320AIC3204}. An input gain function has been implemented in the \gls{dsp} to be able to decrease the output after the effect, if the output is larger than what the \gls{dac} can represent. Thus the requirements is approved.

\subsection{Test of requirement \autoref{req:power2}}
Requirement \autoref{req:power2} states that the product shall be powered by a \SI{9}{\volt} supply. The TMS320C5515 eZdsp development board is powered by a USB 2.0, which is a \SI{5}{\volt} supply. Thus the requirement is not approved.
