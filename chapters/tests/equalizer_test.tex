\section{Test of the equalizer}
In this part, all the requirements related to the equalizer will be tested if they are fulfilled. 


\subsection{Test of requirement \autoref{req:equalizer}}
According to \autoref{req:equalizer}, the gain value in each individual filter must be changeable by the user. This requirement has not been completely fulfilled since there has not been made an actual user interface for the effect. It is possible to change the gain of each individual filter, but only by writing it directly in the program. This is proved in \autoref{}(appendix) \autoref{}(figure), where the peak filter with $\omega_0 =$ \SI{200}{\hertz} and $\omega_0 =$ \SI{3200}{\hertz} is set to amplify respectively \SI{6}{\decibel} and \SI{12}{\decibel}. 




\subsection{Test of requirement \autoref{req:equalizer2}}
According to \autoref{req:equalizer2}, each individual band must, when fully amplified, drop \SI{6}{\decibel} at the neighbouring center frequencies. In \autoref{}(appendix) all the individual, fully amplified, filters and the combined filter, with all filters fully amplified, were tested. In \autoref{} \todo[inline]{Insert figure} it is seen that e.g. the peak filter with $\omega_0 =$ \SI{800}{\hertz} has dropped approximately \SI{6.5}{\decibel} at \SI{400}{\hertz} and \SI{1600}{\hertz}. Thus the requirement is fulfilled. 

